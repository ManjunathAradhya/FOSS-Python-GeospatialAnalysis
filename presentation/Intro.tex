\section[Introduction]{Introduction}
\subsection[Geospatial Data]{Geospatial Data}



\begin{frame}{What is Geospatial data}
	%\Fontvi
	\begin{beamerboxesrounded}{}
		\begin{itemize}
			\item Combination of Geographical and Spatial information 
			\item Management of Geospatial data- Geographical Information System (GIS)
			\item Two data types, Vector and Raster 
			\item Spatial Information denoted by geographical coordinates (Latitude, Longitude) 
			\begin{enumerate}
				\item in 2D- 19.130544,72.916528
				\item in 3D- 19$^{\circ}$7' 49'.9584'' N, 72$^{\circ}$54'59.5008'' E
			\end{enumerate}
			
			%\subitem 
			\item Vector- Point, Line, Polygon
			\item Raster- Bounding box
		\end{itemize}
	\end{beamerboxesrounded}
\end{frame}


%\begin{frame}
%	\only<1-2>{show\_me\_first}
%	\only<2>{show\_me\_second}
%\end{frame}



\begin{frame}{Applications}
	%\Fontvi
	\begin{beamerboxesrounded}{}
		\begin{itemize}
			\item Integral in day to day life
			
			{\centering
			\includegraphics[scale=0.23]{system.png} }
			
			\item Quick Challenge! - Name any two companies 
		\end{itemize}
	\end{beamerboxesrounded}
\end{frame}


\begin{frame}{Application in Environmental or Atmospheric Sciences}
	%\Fontvi
	\begin{beamerboxesrounded}{Some applications}
		\begin{itemize}
			\item Asses global environmental change
			\item Land use/Land cover change  
			\item Pollution emission estimation
			\item Predicting weather and air quality
		\end{itemize}
	\end{beamerboxesrounded}
\end{frame}

\begin{frame}{Typical geospatial analysis and visualisation}

%\onslide<1>{\centering\includegraphics[scale=0.23]{poster.png}} 
%\onslide<2>{\centering\includegraphics[scale=0.6]{pic1a.png}}
%\onslide<3>{\centering\includegraphics[scale=0.18]{pics2.png}}
%\onslide<4>{\centering\includegraphics[scale=0.22]{pics3.png}}
%\onslide<5>{\centering\includegraphics[scale=0.21]{pics4.png}}
%\onslide<6>{\centering\includegraphics[scale=0.21]{pics5.png}}

\only<1>{\centering\includegraphics[scale=0.23]{poster.png}} 
\only<2>{\centering\includegraphics[scale=0.6]{pic1a.png}}
\only<3>{\centering\includegraphics[scale=0.18]{pics2.png}}
\only<4>{\centering\includegraphics[scale=0.22]{pics3.png}}
\only<5>{\centering\includegraphics[scale=0.21]{pics4.png}}
\only<6>{\centering\includegraphics[scale=0.21]{pics5.png}}


\end{frame}

\begin{frame}{Geospatial data property}
	\only<1>{\centering\includegraphics[scale=0.53]{disaster_movie.png}}
\end{frame}

\subsection[Workshop Environment background]{Workshop Environment background}

\begin{frame}{Why Python}
	\only<1>{\centering\includegraphics[scale=0.33]{python.png}}
\end{frame}

\begin{frame}{Geospatial libraries}
	\only<1>{\centering\includegraphics[scale=0.23]{wordcloud.png}}
\end{frame}

\begin{frame}{Jupyter notebooks, JupyterHub, Docker }
	\only<1>{\centering\includegraphics[scale=0.23]{flayer_httyc.png}}
\end{frame}

\begin{frame}{Objective of the workshop}
	%\Fontvi
	\begin{beamerboxesrounded}{}
		To introduce FOSS libraries of Python for Geospatial analysis and visualisation 
	\end{beamerboxesrounded}
\end{frame}



\section[Hello world]{Hello world}
\begin{frame}{On various geospatial data}
	%\Fontvi
	\begin{beamerboxesrounded}{}
		\begin{itemize}
		    \item Neo's vision!
			\item Creation of vectors
			\item Creation of raster
			\item Binary and serial file formats
			\item Gribs, Netcdf
		\end{itemize}
	\end{beamerboxesrounded}
\end{frame}

\section[Vectors]{Vectors}
\begin{frame}{Data analysis and visualization on vectors}
	%\Fontvi
	\begin{beamerboxesrounded}{}
		\begin{itemize}
		    \item Map plotting
			\item Web mapping
			\item Data Analysis
		\end{itemize}
	\end{beamerboxesrounded}
\end{frame}


\section[Rasters]{Rasters}
\begin{frame}{Data analysis and visualization on rasters}
	%\Fontvi
	\begin{beamerboxesrounded}{}
		\begin{itemize}
		    \item Map plotting
			\item Web mapping
			\item Data Analysis
		\end{itemize}
	\end{beamerboxesrounded}
\end{frame}



\begin{frame}{Acknowledgments}
	%\Fontvi
	\begin{beamerboxesrounded}{}
		\begin{itemize}
		    \item Director, Urbanemissions.info
			\item FOSS community
			\item Communtiy behind libraries used in the workshop 
			\item Zero to JupyterHub with Kubernetes
		\end{itemize}
	\end{beamerboxesrounded}
\end{frame}



%\begin{frame}
%	\only<1-2>{show\_me\_first}
%	\only<2>{show\_me\_second}
%\end{frame}