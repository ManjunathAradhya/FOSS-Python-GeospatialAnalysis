% Adapted from: http://web.mit.edu/rsi/www/pdfs/beamer-tutorial.tex

\documentclass[pdf]{beamer}
\mode<all>{\usetheme{Warsaw} \useoutertheme{split}}
\mode<handout>{\usecolortheme{seagull}}
%\usepackage{rsislidepacks}
\usepackage{amsfonts}
\usepackage{amsmath}
\usepackage{amssymb}
\usepackage{amsthm}
\usepackage{graphicx}
\usepackage{url}
%
\usepackage{colortbl}
\usepackage{epstopdf}
\usepackage{eurosym}
\usenavigationsymbolstemplate{}
\usepackage{gensymb}
\newcommand\Fontvi{\fontsize{8}{7.2}\selectfont}
\newcommand\Fontpi{\fontsize{6}{4}\selectfont}

\usepackage{siunitx}
\usepackage{verbatim}
\usepackage{booktabs}
\usepackage{tfrupee}  
\pdfmapfile{=tfrupee.map}
\newcommand{\specialcell}[2][l]{%
	\begin{tabular}[#1]{@{}l@{}}#2\end{tabular}}
\usepackage{multirow}
\setbeamerfont{section in toc}{size=\small}
\setbeamerfont{subsection in toc}{size=\tiny}

\addtobeamertemplate{navigation symbols}{}{%
	\usebeamerfont{footline}%
	\usebeamercolor[fg]{footline}%
	\hspace{1em}%
	\insertframenumber/\inserttotalframenumber
}

%Free and Open Source libraries of Python for all geospatial I/O

\title[\tiny \hspace{145pt} KovaiPy]{Free and Open Source libraries of Python for Geo spatial Analysis and Visualisation(Maps and Satellite imageries)}
\author[Python Tools for Geospatial Analysis]{Nishadh K. A. \newline Research Associate - www.urbanemissions.info \newline @nishadhka }

\begin{document}
	
\AtBeginSection[]{
	\begin{frame}{Table of Contents}
		\tableofcontents[currentsection]
	\end{frame}
}

\begin{frame}
	\thispagestyle{empty}
	\titlepage
\end{frame}
\addtocounter{framenumber}{-1}

\begin{frame}{Table of Contents}
	\tableofcontents
\end{frame}
\section[Introduction]{Introduction}
\subsection[Geospatial Data]{Geospatial Data}


\begin{frame}{What is Geospatial data}
%\Fontvi
\begin{beamerboxesrounded}{}
	\begin{itemize}
		\item Combination of Geographical and Spatial information 
		\item Management of Geospatial data- Geographical Information System (GIS)
		\item Two data types, Vector and Raster 
		\item Spatial Information denoted by coordinates(x,y)(Longitude, Latitude)
		\item Vector- Point, Line, Polygon
		\item Raster- Bounding box
\end{itemize}
\end{beamerboxesrounded}
\end{frame}


\begin{frame}{Typical geospatial analysis}

%\onslide<1>{\centering\includegraphics[scale=0.23]{poster.png}} 
%\onslide<2>{\centering\includegraphics[scale=0.6]{pic1a.png}}
%\onslide<3>{\centering\includegraphics[scale=0.18]{pics2.png}}
%\onslide<4>{\centering\includegraphics[scale=0.22]{pics3.png}}
%\onslide<5>{\centering\includegraphics[scale=0.21]{pics4.png}}
%\onslide<6>{\centering\includegraphics[scale=0.21]{pics5.png}}

\only<1>{\centering\includegraphics[scale=0.23]{poster.png}} 
\only<2>{\centering\includegraphics[scale=0.6]{pic1a.png}}
\only<3>{\centering\includegraphics[scale=0.18]{pics2.png}}
\only<4>{\centering\includegraphics[scale=0.22]{pics3.png}}
\only<5>{\centering\includegraphics[scale=0.21]{pics4.png}}
\only<6>{\centering\includegraphics[scale=0.21]{pics5.png}}


\end{frame}

%\begin{frame}
%	\only<1-2>{show\_me\_first}
%	\only<2>{show\_me\_second}
%\end{frame}



\begin{frame}{Applications}
	%\Fontvi
	\begin{beamerboxesrounded}{}
		\begin{itemize}
			\item Integral in day to day life
			
			{\centering
			\includegraphics[scale=0.23]{system.png} }
			
			\item Quick Challenge! - Name any two companies 
		\end{itemize}
	\end{beamerboxesrounded}
\end{frame}


\begin{frame}{Application in Environmental or Atmospheric Sciences}
	%\Fontvi
	\begin{beamerboxesrounded}{Some applications}
		\begin{itemize}
			\item Asses global environmental change
			\item Land use/Land cover change  
			\item Pollution emission estimation
			\item Predicting weather and air quality
		\end{itemize}
	\end{beamerboxesrounded}
\end{frame}

\subsection[Python programming language]{Python programming language}

\begin{frame}{What is Python}
	%\Fontvi
	\begin{beamerboxesrounded}{}
		"I have used a combination of Perl, Fortran, NCL, Matlab, R and others for routine research, but found out this general- purpose language, Python, can handle almost all in an efficient way from requesting data from remote online sites to statistics, and graphics."
		by a scientist
	\end{beamerboxesrounded}
\end{frame}

\begin{frame}{Python - Characteristics}
	%\Fontvi
	\begin{beamerboxesrounded}{}
		\begin{itemize}
			\item Easy to read and learn
			\item High-level language
			\item General-purpose language
			\item Interpreted language
			\item Extensive libraries
			\item Large userbase 
		\end{itemize}
	\end{beamerboxesrounded}
\end{frame}

\begin{frame}{Anaconda- Python distribution}
	%\Fontvi
	\begin{beamerboxesrounded}{}
		\begin{itemize}
			\item Open source package manager
			\item Virtual environment 
			\item Extensively helped in improve user base
			\item Greater coverage on Data science and machine learning 
		\end{itemize}
	\end{beamerboxesrounded}
\end{frame}

\begin{frame}{Literal programming}
	%\Fontvi
	\begin{beamerboxesrounded}{}
		\begin{itemize}
			\item By Donald Knuth, Stanford University
			\item Readable text with Code section
			\item Well suited for Demonstration, research, and teaching 
			\item A major shift in writing code denoted with thought porcess and background information related to the code 
			\item Jupyter notebooks are for Literal programming
		\end{itemize}
	\end{beamerboxesrounded}
\end{frame}

%\begin{frame}
%	\only<1-2>{show\_me\_first}
%	\only<2>{show\_me\_second}
%\end{frame}
%\section[Component 1]{Component 1: Environment setup}
\subsection[Geographical Information System]{Geographical Information System}

%\Fontvi
\begin{frame}{Geospatial file formats}
	%\Fontvi
	\begin{beamerboxesrounded}{Vector file formate}
		\begin{itemize}
			\item Shape file, a binary 
			\item KML, XML notation 
			\item Geojson, json notation
		\end{itemize}
	\end{beamerboxesrounded}
	\begin{beamerboxesrounded}{Rastor file formate}
		\begin{itemize}
			\item tiff
			\item ascii  
		\end{itemize}
	\end{beamerboxesrounded}
\end{frame}

\begin{frame}{Quantum GIS- QGIS}
		%\Fontvi
		\begin{beamerboxesrounded}{}
			\begin{itemize}
				\item QGIS, cross-platform free and open-source desktop geographic information system (GIS) application
				\item Supports viewing, editing and analysis of geospatial data 
				\item Extensive support on different geospatial data types
				\item Written in C++ with legacy geospatial programs such as GEOS and SQLite. GDAL, GRASS GIS, PostGIS, and PostgreSQL
				\item Plugins as extension 
			\end{itemize}
		\end{beamerboxesrounded}
	\end{frame}
	
	
\begin{frame}{Google earth}
	%\Fontvi
	\begin{beamerboxesrounded}{}
		\begin{itemize}
			\item 3D representation of earth 
			\item Collection of  satellite imagery, aerial photography and GIS data onto a 3D globe
			\item 3D by DEM (Digital Elevation Models)
			\item Origin from Keyhole Earth Viewer
		\end{itemize}
	\end{beamerboxesrounded}
\end{frame}




\subsection[Jupyter notebook]{Jupyter notebook}
\begin{frame}{What are ipynb notebooks}
	%\Fontvi
	\begin{beamerboxesrounded}{}
		\begin{itemize}
			\item A command shell for interactive computing
			\item A browser-based notebook with support for code, text, mathematical expressions, inline plots and other media.
			\item For  introspection, rich media, shell syntax, tab completion, and history.
			\item Language agnostic 
			\item Enhances interactive data visualization and use of GUI toolkits.
			\item Easily shareable 
			\item enables parallel computing 
		\end{itemize}
	\end{beamerboxesrounded}
\end{frame}

\begin{frame}{Markdown language}
	%\Fontvi
	\begin{beamerboxesrounded}{}
		\begin{itemize}
			\item Text component of notebooks
			\item Lightweight markup language
			\item Defacto writing language in web
			\item Simple syntax
		\end{itemize}
	\end{beamerboxesrounded}
\end{frame}


\subsection[Do It Yourself 1]{Do It Yourself 1}
\begin{frame}{DIY 1}
	%\Fontvi
	\begin{beamerboxesrounded}{}
		\begin{itemize}
			\item Install QGIS, Google earth, ice break with geospatial data
			\item Install Anaconda python distribution 
			\item Install the required Jupyter notebook extension
			\item Tasks on Data frame
			\item Tasks on data visualization
		\end{itemize}
	\end{beamerboxesrounded}
\end{frame}


%\section[Component 2]{Component 2: Measure a line}
\subsection[Open street map]{Open street map}


\begin{frame}{Open street map}
	\begin{beamerboxesrounded}{}
		\begin{itemize}
			\item Largest open source web map
			\item Community created source
			\item Nightly build of data in vector file format- shape file format
		\end{itemize}
	\end{beamerboxesrounded}
\end{frame}


\subsection[Python libraries for vector operations]{Python libraries for vector operations}

\begin{frame}{Python libraries for vector operations}
\begin{beamerboxesrounded}{}
	\begin{itemize}
		\item Geopandas to read the shape file
		\item Shapely to perform various vector operation
		\item Rtree index for spatial indexing of large data sources
	\end{itemize}
\end{beamerboxesrounded}
\end{frame}

\subsection{Do It Yourself 2}


\begin{frame}{Do It Yourself 2}
\begin{beamerboxesrounded}{}
	\begin{itemize}
		\item Find distance between two points 
		\item Find distance between two points constrained by another vector \item Find distance between large number of points in for loop	
	\end{itemize}
\end{beamerboxesrounded}
\end{frame}
%\section[Component 3]{Component 3: Search and sort images}
\subsection[Landsat Imagery]{Landsat Imagery}


\begin{frame}{Landsat satellite}
	%\Fontvi
	\begin{beamerboxesrounded}{}
		\begin{itemize}
\item One of the longest satellite program starting from 1970s
\item Total size of imageries from the programme goes above 20 TB
\item Important data source in global land cover change study
\item Data distributed in Geotiff
\end{itemize}
\end{beamerboxesrounded}
\end{frame}

\subsection[Landsat Imagery]{Python libraries for raster operations}

\begin{frame}{Python libraries for raster operations}
%\Fontvi
\begin{beamerboxesrounded}{}
	\begin{itemize}
\item rasterio gives tools to read geotiff
\item Convert the staellite imagery into simple numpy array
\item Several algorithm to clean and detect features in imagery
\end{itemize}
\end{beamerboxesrounded}
\end{frame}

\subsection[Landsat Imagery]{Do It Yourself 3}

\begin{frame}{Do It Yourself 3}
%\Fontvi
\begin{beamerboxesrounded}{}
	\begin{itemize}
\item Convert the imagery in geotiff into numpy arrays 
\item Apply the algorithms to find the cloud cover
\end{itemize}
\end{beamerboxesrounded}
\end{frame}
\end{document}



%total slides=50
%GenralIntro=5
%StudyArea=2
%Chapter1=10
%Chapter2=10
%Chapter3=10
%Chapter4=10
%ConcuFutrDir=3
%GenralIntro
%1-ParticulatePollution
%2-IssuesInterventionMeasures
%3-DataIntensiveApproach
%4-studyObjectives
%5-ThesisOrganization
%StudyArea
%1-CurrentStatus
%2-InterventionMeasures
%Chapter1
%Introcution=1
%Methodology=2
%Results=6
%Conculsion=1
%Chapter2
%Introcution=1
%Methodology=2
%Results=6
%Conculsion=1
%Chapter3
%Introcution=1
%Methodology=2
%Results=6
%Conculsion=1
%Chapter4
%Introcution=1
%Methodology=2
%Results=6
%Conculsion=1
%ConcuFutrDir=3
